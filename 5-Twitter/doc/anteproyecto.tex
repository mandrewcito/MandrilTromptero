
% Usar el tipo de documento: Artículo científico.
\documentclass[12pt,a4paper]{article}

% Cargar mensajes en español.
\usepackage[spanish]{babel}

% Usar codificación utf-8 para acentos y otros.
\usepackage[utf8]{inputenc}


% Comenzar párrafos con separación no indentación
\usepackage{parskip}

% Propiedades
\title{Analizador de la evolución de opiniones en Twitter}

\author{Andrés Baamonde Lozano (andres.baamonde@udc.es)}
	
\begin{document}
\maketitle

%%%%%%%%%%%%%%%%%%%%%%%%%%%%%%%%%%%%%%%%%%%%%%%%%%%%%%%%%%%%%%%%%%%%%%%%%%%%%%%

\section{Objetivo}
Obtener a lo largo del tiempo tweets sobre un determinado tema/persona/producto/evento, y analizar la evolución temporal de las opiniones que se expresan en Twitter sobre ese tema, agregándola y proporcionando una representación gráfica.
\section{Descripción}
Twitter es una de las redes sociales más utilizadas en el mundo. Los medios de comunicación, las compañías, etc.. Se apoyan en esta red social para conocer la opinión que uno de sus programas/productos que no es, ni mucho menos, representativa de la totalidad de la población pero, al ser una parte tan grande, tampoco podemos obviar.Esta página permitirá saber como evoluciona a lo largo del tiempo la opinión, también, si existe algo que aparte de nuestro producto/tema que mencionen frecuentemente esos tweets, sabiendo si es una sentencia mala o buena y posteriormente se analizaran los eventos que pudieron dar lugar al cambio en la opinión general de un sector.Se accederá a twitter por medio de la librería tweepy de python, almacenaremos los tweets en una base de datos documental, cada tweet sera preprocesado y posteriormente analizado con miopia.
\section{Metodología}
Se utilizará la metodología scrum que proporcionará un resultado completo en  cada incremento.
\section{Fases}
\begin{itemize}
\item Familiarización con las tecnologías a emplear a lo largo del proyecto.
\item Análisis y diseño de herramienta para rellenar BD y aplicación web.
\item Implementación.
\item Prueba aplicación.
\end{itemize}
\section{Material}
Un PC para el desarrollo de la aplicación.
\end{document}